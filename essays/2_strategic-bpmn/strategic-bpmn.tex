\documentclass[14pt,a4paper]{article}
\usepackage[utf8]{inputenc}
\usepackage[T1]{fontenc}
\usepackage{amsmath}
\usepackage{amsfonts}
\usepackage{amssymb}
\usepackage{graphicx}
\usepackage{hyperref}
\usepackage[style=authoryear,backend=biber]{biblatex}
\DeclareDelimFormat{postnotedelim}{\addcomma\space}
\usepackage{csquotes}
\usepackage[margin=3.5cm]{geometry}
\usepackage{titlesec}
\usepackage{appendix}
\usepackage{booktabs}
\usepackage{longtable}
\usepackage{fancyhdr}
\usepackage{xurl}

\addbibresource{references.bib}

\titleformat{\section}
  {\normalfont\Large\bfseries}{\thesection}{1em}{}
\titleformat{\subsection}
  {\normalfont\large\bfseries}{\thesubsection}{1em}{}

\title{Strategic Business Process Modelling: Transforming i* Socio-Technical Requirements into BPMN Representations for Automotive Repair Services}
\author{Enzo Joly, 22055453}
\date{}

\renewcommand{\labelitemi}{-}

\pagestyle{fancy}
\fancyhf{}
\renewcommand{\headrulewidth}{0.4pt}
\renewcommand{\footrulewidth}{0.4pt}
\fancyhead[L]{UFCFAF-30-3 | Development of Information Systems Projects}
\fancyhead[R]{Page \thepage}
\fancyfoot[C]{\thepage}

\begin{document}

\maketitle

\hrule

\vspace{3em}

Word Count: 1000

\vspace{3em}
\hrule

\vspace{2em}
\textbf{Abstract}
\vspace{1em}

This paper presents a strategic business process model for automotive repair services, translating socio-technical i* requirements into coherent BPMN representations. Through methodical translation of actor dependencies and intentional elements into process flows, the research addresses how strategic-level process modelling effectively bridges the gap between stakeholder goals and operational execution. By focusing on critical process touchpoints, participant interactions, and decision gateways, the model prioritises visibility of strategic elements whilst maintaining comprehensibility. The resulting BPMN model provides a foundation for executive comprehension and subsequent operational elaboration, demonstrating how process abstraction techniques enhance communication effectiveness for managerial stakeholders whilst preserving essential socio-technical elements captured in goal-oriented requirements engineering.

\vspace{3em}
\hrule

\thispagestyle{empty}

\newpage

\tableofcontents
\pagenumbering{roman}

\newpage

\pagenumbering{arabic}

\section{Introduction}

The transition from requirements analysis to process implementation represents a significant challenge in information systems development. Strategic business process models serve as essential intermediaries between high-level socio-technical models and operational implementations, providing what \textit{\parencite[p. 76]{Silver2011}} describes as the "crucial abstraction layer" that frames enterprise processes. Building upon the previously established i* socio-technical model of automotive repair services, this paper constructs a strategic Business Process Model and Notation (BPMN) representation of the repair service ecosystem. This translation exemplifies what \textit{\parencite[p. 33]{Poels2018}} characterises as "the conceptual bridge linking organisational goals with operational activities," forming a vital link in the requirements engineering chain.

Strategic process models serve distinct purposes from their operational counterparts, focusing on executive comprehension rather than implementation details. As \textit{\parencite[p. 152]{Dijkman2011}} notes, "strategic models prioritise communication efficacy over execution specificity," making them particularly valuable for managerial stakeholders. For automotive repair services, where complex dependencies exist between customers, reception staff, mechanics, and support services, a strategic BPMN model visualises essential service flows whilst abstracting implementation complexities.

\section{Theoretical Foundations}

Strategic BPMN modelling occupies a distinctive position in the business process hierarchy. According to \textit{\parencite[p. 17]{Allweyer2016}}, strategic models "provide the high-level process architecture within which operational concerns exist," focusing on what \textit{\parencite[p. 89]{Dumas2018}} terms "core process patterns rather than execution variations." This abstraction serves multiple purposes, including executive communication, process boundary definition, and architectural framing.

The translation from goal-oriented requirements engineering (GORE) to strategic process models represents a methodological shift from intentional modelling to process flow. As \textit{\parencite[p. 342]{Dalpiaz2016}} observe, this transformation "converts actor dependencies and rationales into sequential activities with defined handoffs," while preserving essential socio-technical considerations. This preservation is particularly crucial for service-oriented domains like automotive repair, where customer experience and trust relationships remain central to business success.

\section{Methodology}

The construction of the strategic BPMN model employed a systematic transformation approach based on the i* socio-technical framework previously developed. Following the methodology advocated by \textit{\parencite[p. 208]{Cardoso2011}}, the translation process comprised three distinct phases: (1) actor mapping to pools and lanes, (2) dependency conversion to message flows, and (3) task and goal transformation into activities and events.

This approach preserved critical socio-technical elements whilst adhering to the strategic modelling principles identified by \textit{\parencite[p. 177]{Silver2011}}: maintaining comprehensibility, limiting diagram complexity, and focusing on the "happy path" process flow. In alignment with \textit{\parencite[p. 92]{Flowers2016}}, the modelling procedure specifically abstracted exceptional flows and detailed execution logic to maintain strategic clarity.

The resulting model underwent validation using the quality criteria framework developed by \textit{\parencite[p. 136]{Corradini2018}}—focusing on syntactic correctness, semantic adequacy, and pragmatic effectiveness—to ensure it effectively communicated the automotive repair process at the strategic level.

\section{Strategic Model Elements and Translation}

\subsection{Actor Translation and Pool Structure}

The i* model identified four primary actors: Customer, Receptionist, Mechanical Team, and Towing Service. These actors were directly mapped to BPMN pools, establishing clear process boundaries and responsibilities. Following \textit{\parencite[p. 103]{Kluza2017}}' guidance, "pools in strategic models should represent primary organisational units or external stakeholders," with lanes used sparingly to avoid excessive vertical complexity.

The customer pool assumes particular significance, representing what \textit{\parencite[p. 217]{Dumas2018}} identify as the "process initiator and ultimate value recipient." The translation maintained role clarity whilst avoiding what \textit{\parencite[p. 67]{Geiger2018}} caution against as "excessive role fragmentation that obscures process flow."

\subsection{Strategic Activities and Elements}

Core activities within the strategic model directly correspond to critical dependency relationships identified in the i* framework. Key elements include:

\begin{enumerate}
    \item \textbf{Initial Contact and Service Negotiation} - Translating customer goals ("Get Vehicle Repaired") and receptionist tasks ("Manage Customer Expectations") into coordinated process activities.

    \item \textbf{Membership Programme Integration} - Capturing membership dependencies that \textit{\parencite[p. 263]{Castro2018}} characterised as a "strategic feedback loop" through dedicated activities for membership verification and discount application.

    \item \textbf{Diagnostic and Repair Process} - Representing the mechanical team's goal dependencies ("Maintain Repair Quality") and resource dependencies ("Diagnostic Results") through sequential activities with appropriate message flows.

    \item \textbf{Quality Verification and Delivery} - Modelling the completion dependencies identified in the i* model through verification tasks and milestone events.
\end{enumerate}

As \textit{\parencite[p. 158]{Koschmider2019}} observe, strategic models should "focus on the what rather than the how," emphasising business outcomes over implementation techniques. This principle guided the abstraction level of all activities within the model.

\subsection{Strategic Gateways and Decision Points}

Key decision points were modelled using exclusive gateways, focusing on what \textit{\parencite[p. 185]{Allweyer2016}} identify as "business-critical divergence points" rather than operational variations. Following guidance from \textit{\parencite[p. 42]{Bocciarelli2017}}, the model employs minimal gateway complexity, using them only "where strategic alternatives exist that change the fundamental process flow."

Notable gateways include the membership status verification, repair requirements assessment, and customer approval decision points—directly corresponding to the i* model's alternative means-ends relationships and softgoal contribution links. This selective use of gateways maintains strategic clarity while preserving essential decision visibility.

\section{Strategic Implications and Analysis}

The strategic BPMN model reveals significant insights that complement the i* analysis whilst providing additional process perspective. Primary findings include:

\subsection{Process Boundaries and Integration Points}

The model clarifies process boundaries between organisational units, highlighting what \textit{\parencite[p. 219]{Kluza2017}} term "critical handoff points requiring careful integration." The coordination between reception, mechanical services, and towing represents a particular focus area, with message flows indicating what \textit{\parencite[p. 48]{Gorton2017}} identified as "potential friction points requiring careful information system mediation."

\subsection{Strategic Feedback and Monitoring Points}

The model establishes clear monitoring points where process effectiveness can be evaluated, corresponding to what \textit{\parencite[p. 195]{Delgado2020}} describe as "critical measurement opportunities for process optimisation." These include service completion notification, customer approval, and quality verification steps—each representing opportunities for metrics collection and performance evaluation.

\subsection{Customer Engagement and Communication}

Customer engagement patterns become explicitly visible in the strategic model, showing when and how customers participate in the process. This addresses what \textit{\parencite[p. 177]{Estrada2020}} identified as "multi-dimensional strategic instruments" in the i* model, particularly regarding trust-building mechanisms and transparency processes. The strategic BPMN model makes these communication dependencies explicit through clearly defined message flows and interaction points.

\section{Conclusion}

The translation from i* socio-technical models to strategic BPMN representations provides essential architectural framing for the automotive repair service process. This strategic model maintains focus on critical business elements while abstracting operational complexities, creating what \textit{\parencite[p. 103]{Bosch2018}} describe as "the essential communication bridge between executive vision and operational implementation."

The strategic model reveals process patterns that confirm critical findings from the i* analysis, particularly regarding customer trust dependencies, quality-efficiency tensions, and membership programme integration. By explicitly modelling message flows and coordination points, the BPMN representation enhances understanding of process interactions whilst maintaining the socio-technical perspective.

As the foundation for subsequent operational elaboration, this strategic model establishes process boundaries, identifies integration requirements, and highlights measurement opportunities—all whilst maintaining comprehensibility for managerial stakeholders. Through this deliberate abstraction, the strategic BPMN model fulfils its essential purpose as a communication tool that frames implementation decisions within the broader organisational context of the automotive repair service.

\newpage

\printbibliography

\end{document}
