\documentclass[14pt,a4paper]{article}
\usepackage[utf8]{inputenc}
\usepackage[T1]{fontenc}
\usepackage{amsmath}
\usepackage{amsfonts}
\usepackage{amssymb}
\usepackage{graphicx}
\usepackage{hyperref}
\usepackage[style=authoryear,backend=biber]{biblatex}
\DeclareDelimFormat{postnotedelim}{\addcomma\space}
\usepackage{csquotes}
\usepackage[margin=3.5cm]{geometry}
\usepackage{titlesec}
\usepackage{appendix}
\usepackage{booktabs}
\usepackage{longtable}
\usepackage{fancyhdr}
\usepackage{xurl}

\addbibresource{references.bib}

\titleformat{\section}
  {\normalfont\Large\bfseries}{\thesection}{1em}{}
\titleformat{\subsection}
  {\normalfont\large\bfseries}{\thesubsection}{1em}{}

\title{Operational Business Process Modelling for Automotive Repair Services: BPMN Implementation with User Interface Integration}
\author{Enzo Joly, 22055453}
\date{}

\renewcommand{\labelitemi}{-}

\pagestyle{fancy}
\fancyhf{}
\renewcommand{\headrulewidth}{0.4pt}
\renewcommand{\footrulewidth}{0.4pt}
\fancyhead[L]{UFCFAF-30-3 | Development of Information Systems Projects}
\fancyhead[R]{Page \thepage}
\fancyfoot[C]{\thepage}

\begin{document}

\maketitle

\hrule

\vspace{3em}

Word Count: 1000

\vspace{3em}
\hrule

\vspace{2em}
\textbf{Abstract}
\vspace{1em}

This paper presents an operational business process model for automotive repair services implemented in BPMN 2.0, transitioning from strategic to executable process specifications. The translation elaborates abstract activities into detailed sequences with concrete actor interactions, focusing particularly on user interface integration. Through systematic disaggregation of high-level processes and implementation of user forms at critical interaction points, the resulting model establishes an executable foundation for process automation. The analysis specifically emphasises form design considerations, data collection requirements, and validation rules whilst maintaining alignment with the socio-technical requirements identified in earlier modelling phases. This operational BPMN model delivers the necessary specificity for direct implementation within a process automation engine whilst preserving essential business objectives and actor relationships established in previous goal-oriented and strategic modelling efforts.

\vspace{3em}
\hrule

\thispagestyle{empty}

\newpage

\tableofcontents
\pagenumbering{roman}

\newpage

\pagenumbering{arabic}

\section{Introduction}

The transition from strategic to operational business process modelling represents a critical phase in information systems development. Where strategic models provide high-level process architecture focusing on what \textit{\parencite[p. 213]{Allweyer2016}} describes as "executive comprehension and process boundaries," operational models establish the concrete foundation for automation through detailed activity specification and user interface integration. Building upon the previously developed i* socio-technical model and strategic BPMN representation, this paper presents a comprehensive operational model that transforms conceptual processes into executable specifications.

Operational BPMN models serve fundamentally different purposes from their strategic counterparts. As \textit{\parencite[p. 178]{Dumas2018}} observe, "operational models provide the necessary implementation detail to support direct execution in workflow engines," including exception handling, data requirements, and user interface specification. For automotive repair services—where customer interaction, service coordination, and quality control are paramount—the operational model must specify exactly how these interactions occur whilst enabling process automation.

The distinction between strategic and operational perspectives is particularly salient in user interaction design. According to \textit{\parencite[p. 62]{Peffers2018}}, "operational models must specify precisely where and how humans interact with the process," explicitly identifying form requirements, data validation, and human-system boundaries. This paper specifically emphasises these interaction points, establishing a foundation for complete process automation through designed user interfaces at critical touchpoints.

\section{Theoretical Foundations of Operational Process Modelling}

Operational business process models transform abstract activities into executable sequences through what \textit{\parencite[p. 329]{Weske2019}} characterises as "progressive detailing of process steps, incorporating technical concerns whilst maintaining business intent." This transition necessitates integrating theoretical elements from both workflow management and user interface design disciplines.

\textit{\parencite[p. 143]{Rodriguez-Dominguez2022}} identify three critical dimensions that differentiate operational from strategic models: "(1) comprehensive exception handling, (2) explicit data requirements, and (3) detailed user interface integration." These dimensions establish what \textit{\parencite[p. 92]{LaCruz2021}} term the "implementation boundary" where business processes intersect with technical systems.

For service-oriented domains like automotive repair, operational modelling demands particular attention to what \textit{\parencite[p. 217]{Dumas2018}} call "moments of truth"—customer interaction points where service quality is directly experienced. These moments establish the critical form requirements that shape automation design, determining where and how user input integrates with process flows.

\section{Methodology}

The construction of the operational BPMN model employed a systematic elaboration methodology based on the strategic model previously developed. Following the approach advocated by \textit{\parencite[p. 117]{Schönig2018}}, the transformation comprised four sequential phases: (1) activity decomposition into detailed task sequences, (2) exception path identification, (3) data object specification, and (4) user interface identification.

This approach aligns with the "progressive elaboration" methodology recommended by \textit{\parencite[p. 195]{Marin2018}} for transitioning from strategic to operational models. Each strategic activity was systematically analysed for sub-activities, interaction requirements, and data dependencies, establishing what \textit{\parencite[p. 76]{Silver2011}} describes as "implementation completeness" necessary for automation.

User form requirements were identified using the "interaction point analysis" technique proposed by \textit{\parencite[p. 112]{Yousfi2019}}, systematically examining each human-system interaction to determine required data collection, validation rules, and user experience considerations. This analysis specifically incorporated the trust relationships and knowledge boundaries identified in the i* model, ensuring these socio-technical concerns were addressed in the form design.

\section{Operational Model Elements}

\subsection{Pools, Lanes, and Activity Specification}

The operational model expands upon the strategic representation by maintaining the four primary actor pools—Customer, Receptionist, Mechanical Team, and Towing Service—whilst introducing detailed activity sequences within each. Following \textit{\parencite[p. 147]{Kluza2017}}' guidance, "operational lanes should reflect functional responsibilities within organisational units," the model implements additional segregation in the mechanical pool to distinguish diagnostic and repair responsibilities.

Activity sequences are detailed to what \textit{\parencite[p. 83]{Meyer2019}} term "atomic execution units"—individual tasks that cannot be meaningfully subdivided further. This granularity establishes clear process visibility whilst enabling direct implementation in automation tools. Critical sequences include:

\begin{enumerate}
    \item \textbf{Customer Engagement Flow} – From initial contact through service approval to payment and feedback, with form integration at each interaction point.

    \item \textbf{Reception Process} – Detailed activities for customer information capture, membership verification, service scheduling, and payment processing with corresponding form specifications.

    \item \textbf{Mechanical Procedures} – Systematic decomposition of diagnostic and repair activities with quality verification touchpoints and customer approval mechanisms.

    \item \textbf{Towing Integration} – Specific activities for breakdown reporting, vehicle collection, and service handoff with corresponding data requirements.
\end{enumerate}

Each activity sequence includes explicit start/end events, intermediate message events for coordination, and timer events for service level monitoring—establishing a complete control flow that \textit{\parencite[p. 259]{vanderAalst2016}} identify as essential for "operational process execution in automation engines."

\subsection{Gateway Logic and Exception Handling}

The operational model implements comprehensive decision logic through explicit gateway conditions reflecting business rules and exception scenarios. Following \textit{\parencite[p. 117]{Corradini2018}}' recommendation for "operational decision clarity," gateway conditions are specified with precise evaluation criteria referencing specific data objects.

Major decision points include:

\begin{enumerate}
    \item \textbf{Service Type Determination} – Exclusive gateways branching between standard service, emergency towing, and warranty processes.

    \item \textbf{Membership Verification} – Conditional flow determination based on membership status with corresponding discount application.

    \item \textbf{Repair Approval Flow} – Customer decision gateway with approval and rejection paths, including rework loops for rejected repairs.

    \item \textbf{Quality Verification} – Testing gateway ensuring repair quality before customer delivery, with rework paths for failed verification.
\end{enumerate}

Exception handling is implemented through what \textit{\parencite[p. 136]{Meyer2019}} identify as "compensating paths"—dedicated flows triggered when normal process execution cannot proceed. These include payment failure handling, service cancellation procedures, and quality remediation processes—ensuring the model captures not only the "happy path" but also essential exception scenarios.

\subsection{User Form Integration and Data Requirements}

The operational model explicitly identifies user interface integration points through thirteen form specifications linked to human interaction activities. Following \textit{\parencite[p. 167]{Pulvermueller2022}}' "form-task integration framework," each form specification includes:

\begin{enumerate}
    \item \textbf{Customer Information Form} – Capturing vehicle details, contact information, and service requirements with validation rules for mandatory fields and format restrictions.

    \item \textbf{Membership Registration Form} – Collecting subscription details with terms acceptance and payment information, including validation for required fields and duplicate prevention.

    \item \textbf{Diagnostic Authorisation Form} – Securing customer approval for diagnostic procedures with estimated costs and terms explanation.

    \item \textbf{Repair Approval Form} – Presenting diagnostic results and repair recommendations with itemised cost estimates and authorisation controls.

    \item \textbf{Service Feedback Form} – Collecting customer satisfaction data and improvement suggestions through structured rating scales and comment fields.
\end{enumerate}

Each form specification directly addresses what \textit{\parencite[p. 76]{Rifaut2020}} identified as "critical service interaction points" in the i* model, implementing the "transparent service verification mechanisms" through structured information exchange. Form designs incorporate validation rules ensuring data integrity whilst supporting the trust relationships identified in earlier modelling phases.

\section{Implementation Considerations}

The operational model establishes a comprehensive foundation for direct implementation in process automation platforms. According to \textit{\parencite[p. 227]{Burns2023}}' "implementation readiness criteria," the model provides sufficient detail for technical deployment through:

\begin{enumerate}
    \item \textbf{Complete Control Flow Specification} – Fully defined activity sequences, gateway conditions, and event triggers supporting automated routing.

    \item \textbf{Comprehensive Data Model} – Explicit data objects with attributes, relationships, and persistence requirements that establish automation data structures.

    \item \textbf{Specified Human Interactions} – Clearly defined user interface requirements with form specifications at all human touchpoints.

    \item \textbf{Integration Requirements} – Identified system boundaries and external service connections supporting technical integration.
\end{enumerate}

This implementation foundation addresses what \textit{\parencite[p. 47]{Lezcano2022}} identified as the need for "alignment with organisational motivations" by maintaining clear connections to the strategic process goals whilst providing executable specificity.

\section{Conclusion}

The transition from strategic to operational BPMN modelling for automotive repair services demonstrates the progressive elaboration necessary for implementing complex socio-technical systems. Through systematic decomposition of abstract activities into executable sequences, integration of user interface specifications, and implementation of business rules through gateway logic, the operational model transforms conceptual processes into automation-ready specifications.

The model maintains essential socio-technical considerations identified in the i* framework, particularly regarding trust relationships, knowledge boundaries, and quality-efficiency tensions. By implementing these concerns through specific form designs, validation rules, and approval mechanisms, the operational model preserves human values within technical implementation—addressing what \textit{\parencite[p. 217]{Dumas2018}} identify as the essential "balance between automation efficiency and human-centred service quality."

As the foundation for process automation, this operational model delivers the specificity required for technological implementation whilst preserving alignment with business objectives. The systematic integration of user forms at critical interaction points ensures that automation enhances rather than replaces essential human judgement, establishing a foundation for what \textit{\parencite[p. 197]{Calegari2020}} describe as "socio-technically balanced process automation" that supports rather than supplants human capabilities.

\newpage

\printbibliography

\end{document}
